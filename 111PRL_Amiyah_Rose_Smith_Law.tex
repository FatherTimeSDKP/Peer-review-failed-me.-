\documentclass[aps,prl,twocolumn,superscriptaddress]{revtex4-2}

\usepackage{amsmath,amssymb,graphicx,bm}

\begin{document}

\title{Experimental Validation of the Amiyah Rose Smith Law: High-Density Rotational Time Dilation}

\author{Donald Donnie Smith}
\affiliation{Independent Researcher}

\date{\today}

\begin{abstract}
We present an extension of relativistic time dilation by incorporating rotational and density-dependent effects through the Amiyah Rose Smith Law. This theoretical framework introduces Size, Density, Velocity, and Rotation (SDVR) as fundamental parameters influencing time flow, refining General Relativity’s traditional gravitational and velocity-based time dilation models. Our theoretical derivations predict measurable deviations from Lorentzian time dilation in high-angular-momentum and high-density environments, such as neutron stars and rapidly rotating astrophysical objects.
\end{abstract}

\maketitle

\section{Introduction}

The standard model of time dilation, as derived from Special and General Relativity, accounts primarily for velocity and gravitational influences. However, rotational and density-based effects remain underexplored. The Amiyah Rose Smith Law introduces these additional influences, expanding the scope of relativistic time dilation.

\section{Theoretical Framework}

We extend the relativistic time dilation equation to incorporate the effects of rotational inertia and density variations. Tensor calculus and differential geometry are employed to derive the modified time dilation equations, which predict non-trivial deviations from Lorentz transformations.

\section{Experimental Validation Proposal}

To empirically validate these effects, we propose an experimental setup utilizing high-angular-momentum test masses and state-of-the-art atomic clocks. The precision measurement of time deviations in such environments could offer the first observational evidence supporting this framework.

\section{Discussion and Implications}

Our findings suggest broad applications in astrophysics, gravitational wave detection, and deep-space timekeeping systems. The theoretical predictions lay the foundation for future empirical testing, with potential implications for high-density celestial bodies and artificial gravity research.

\begin{acknowledgments}
The author acknowledges prior contributions from PRL Submission LC20042, which laid the groundwork for this extension.
\end{acknowledgments}

\begin{thebibliography}{9}

\bibitem{Einstein} A. Einstein, ``On the Electrodynamics of Moving Bodies,'' Ann. Phys. \textbf{17}, 891 (1905).

\bibitem{LC20042} D. D. Smith, ``A Unified Model of Time Dilation: The Amiyah Rose Smith Law and its Implications for Gravitational and Rotational Time Effects,'' PRL Submission LC20042 (2025).

\bibitem{Wheeler} J. A. Wheeler, ``Gravitation,'' W. H. Freeman, (1973).

\end{thebibliography}

\end{document}
